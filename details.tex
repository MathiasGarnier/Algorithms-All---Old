	\documentclass[10pt]{article}
	\usepackage[a4paper,bottom = 0.6in,left = 0.75in,right = 0.75in,top = 1cm]{geometry}
	\usepackage{graphicx}
	\usepackage{amsmath}
	\usepackage{array}
	\usepackage{enumitem}
	\usepackage{wrapfig}
	\usepackage{microtype}
	\usepackage{titlesec}
	\usepackage{textcomp}
	\usepackage[colorlinks=false]{hyperref}
	\usepackage{verbatim}
	\usepackage{makecell}
	\usepackage{pbox}
	\usepackage{hyperref}
	\usepackage{tabularx}
	\usepackage{amsfonts}
	
	%for color
	\usepackage[usenames, dvipsnames]{color}
	\definecolor{myblue}{RGB}{76,81,150}
	\definecolor{gray}{RGB}{110, 87, 86}
	
	\newcommand{\xfilll}[2][1ex]{
	\dimen0=#2\advance\dimen0 by #1
	\leaders\hrule height \dimen0 depth -#1\hfill}
	\titleformat{\section}{\large\scshape\raggedright}{}{0em}{}
	\renewcommand\labelitemi{\raisebox{0.4ex}{\tiny$\bullet$}}
	\renewcommand{\labelitemii}{$\cdot$}
	\pagenumbering{gobble}
	\newcommand\tab[1][1cm]{\hspace*{#1}}
	
	\hypersetup{
	    colorlinks=true,
	    urlcolor=gray,
	}
	
	
	
	
	
	
	
	\begin{document}
	
		\begin{center}
	
		        \textit{Rappel. Ceci n'est pas un CV public.}\\
				{\Large \textbf{GARNIER Mathias}\par}
				Naissance : 4 juin 2002, Lyon \\
				%\href{mailto:mathias.garnier@ut-capitole.fr}{mathias.garnier@ut-capitole.fr} \\
				%\href{mailto:droitavoirdesinfos.mathiasg@gmail.com}{droitavoirdesinfos.mathiasg@gmail.com}  \\
				\href{mailto:mathias.garnier69000@gmail.com}{mathias.garnier69000@gmail.com} \\
				(+33) 6 45 67 94 31
		\end{center}
	
	
		
	\vspace{2mm}
	\section*{{\LARGE \color{myblue}Scolarité et études}\xfilll[0pt]{0.5pt}}
	\vspace{-15pt}
	\vspace{1.5mm}
	
	
	
	\begin{itemize}[itemsep = -0.70 mm, leftmargin=*]
		\item[--] \noindent Licence - Double diplôme - \textbf{Licence Mathématiques} et D.U. \textbf{Parcours Spéciaux}, Université Paul Sabatier, Toulouse. \hfill {\ \small [2022 -- XXXX]}
		\item[--] \noindent \textbf{Droit}, parcours double diplôme international franco-italien (Milan), Université Toulouse 1 Capitole, \textit{European School of Law} (\textbf{ESL}). \hfill {\ \small [2020 -- 2022]}
		\item[--] \noindent \textbf{Lycée Ozenne}, Toulouse. Section scientifique, spécialité Mathématiques. \hfill {\ \small [2017 -- 2020]}
	\end{itemize}	
	
	
	
	
	\vspace{2mm}
	\section*{{\LARGE \color{myblue}Stages et formations complémentaires}\xfilll[0pt]{0.5pt}}
	\vspace{-15pt}
	\vspace{1.5mm}
	
	
	\begin{itemize}[itemsep = -0.70 mm, leftmargin=*]
		\item[--] \noindent \textbf{Stage d'introduction à la chimie quantique computationnelle} et à la programmation scientifique sous la dir. de A. Scemama, Laboratoire de Chimie et Physique Quantique, \textbf{CNRS}, UT3. \hfill {\ \small [2023]}	
		\item[--] \noindent \textbf{Introduction à la recherche}, \textbf{projet de Mathématiques} sur l'approximation uniforme sous la direction de Philippe Monnier avec Garcia Hugo et Sala Raphaël. \hfill {\ \small [2023]}
		\item[--] \noindent Aide à l'\textbf{édition} de documents mathématiques chez \textbf{Calvage\&Mounet} (sous la dir. de R. Mneimné). \hfill {\ \small [2023]}
		\item[--] \noindent \textbf{Formation} de l'Agence Nationale de la Sécurité des Systèmes d'Information (\textbf{ANSSI}), MOOC. \hfill {\ \small [2022]}
		\item[--] \noindent Parcours annuel, \textbf{Académie Notre Europe}, Institut Jacques Delors. Démissionnaire. \hfill {\ \small [2021 -- 2022]}
		\item[--] \noindent \textbf{Activité de préservation et de valorisation de l'héritage culturel} d'Idrija (Slovénie) (Corps européen de solidarité, 3 semaines). \hfill {\ \small [2021]}
		\item[--] \noindent Bourse \textbf{Zellidja} de voyage et d'étude en Italie sur le thème \textit{Dante Alighieri, d'une oeuvre à quelques unes de ses traces}. \hfill {\ \small [2019]}
		\item[--] \noindent Camp d'été, \textbf{Centre International de Mathématiques et d'Informatique}, Université Paul Sabatier, Toulouse. \hfill {\ \small [2019]}
		\item[--] \noindent Stage d'observation en milieu professionel, \textbf{Laboratoire de Chimie et Physique Quantique}, CNRS, Université Paul Sabatier, Toulouse. \hfill {\ \small [2016]}
	\end{itemize}	
	
	
	\vspace{5mm}
	\vspace{-20pt}
	\section*{{\LARGE \color{myblue}Activités associatives et bénévoles}\xfilll[0pt]{0.5pt}}
	\vspace{-15pt}
	\vspace{1.5mm}
	
	
	
	\begin{itemize}[itemsep = -0.70 mm, leftmargin=*]
		\item[--] \noindent Membre du \textit{Corps européen de solidarité} (\textbf{European Solidarity Corps} under the supervision of the European Commission). \hfill {\ \small [2021 -- 2032]}
		\item[--] \noindent Membre cotisant de la \textbf{Société Mathématique de France}. \hfill {\ \small [2021]}
		\item[--] \noindent Utilisateur de \textbf{Transcribathon} (une centaine de documents transcrits, 300'000 caractères). \hfill {\ \small [2021]}
		\item[--] \noindent Bénévole comité de lecture éditions \textbf{Racaille} / \textbf{Zellidja}. \hfill {\ \small [2021]}
		\item[--] \noindent Bénévole \textbf{Zellidja}, pôle \textit{Vie associative et démocratie}. \hfill {\ \small [2020 -- 2021]}
		\item[--] \noindent \textbf{Soutien scolaire} bénévole en Mathématiques (Lycée, Licence 1...). \hfill {\ \small [2019 -- 2022]}
		\item[--] \noindent \textbf{Délégué de classe}, Terminale S (Lycée Ozenne). \hfill {\ \small [2019 -- 2020]}
		\item[--] \noindent Lecture au \textbf{Théâtre National de Toulouse} (\textit{Les valises-lecture}). \hfill {\ \small [2014]}
		\item[--] \noindent \textbf{Suppléant de classe}, 6ème (Collège Les Chalets). \hfill {\ \small [2013 -- 2014]}
	\end{itemize}	
	
	
	
	%\vspace{5mm}
	%\vspace{-20pt}
	%\section*{{\LARGE \color{myblue}Activités parallèles et formations additionnelles}\xfilll[0pt]{0.5pt}}
	%\vspace{-15pt}
	%\vspace{1.5mm}
	
	
	
	
	
	%	\item[--] \noindent PROJET INTERTNATIONAL DE MATHS ETUDIANTS A REDIGER PLUS TARD. \hfill {\ \small [Octobre 2021 -- 2023]}			-> A VOIR COMMENT LE PRESENTER
	%	\item[--] \noindent Ateliers bihebdomadaires de \textbf{programmation}. \hfill {\ \small [2021 -- 2022]}
	%	\item[--] \noindent Auditeur libre \textbf{Collège de France}. \hfill {\ \small [2021 -- Aujourd'hui]}
	%	\item[--] \noindent Projet d'\textbf{apprentissage des mathématiques en autodidacte}.			-> INUTILE DE LE METTRE SI JE PARS EN MATHS
		%\\ \tab Comptes rendus réalisés chaque mois : \href{http://www.les-mathematiques.net/phorum/read.php?32,2229954,2254902#msg-2254902}{[1]}. 
	%	\hfill {\ \small [2021 -- Aujourd'hui]}
	%	\item[--] \noindent Utilisateur de \textbf{ProjectEuler} (15 problèmes résolus). \hfill {\ \small [2021 - Aujourd'hui]}
	
	
	
	
	
	\vspace{5mm}
	\vspace{-20pt}
	\section*{{\LARGE \color{myblue}Bagages auxiliaires}\xfilll[0pt]{0.5pt}}
	\vspace{-15pt}
	\vspace{1.5mm}
	
	
	
	\begin{itemize}[itemsep = -0.70 mm, leftmargin=*]
		\item[--] \noindent \textbf{Programmation :} Python, C, (Fortran), \LaTeX.
		\item[--] \noindent \textbf{Langues vivantes :} français (langue maternelle), italien (lu, écrit, parlé), anglais (lu, écrit, quelque peu parlé); allemand (deux échanges scolaires, pour un total de trois semaines).
		\item[--] \noindent \textbf{Langues mortes :} latin (débutant), grec ancien (une année).
	%	\item[--] \noindent \textbf{Sport : } football (cinq années), tennis (cinq années), badminton (quatre années en structure associative; une année en universitaire), aviron (une année); pratique libre course à pied et cyclisme. 
		\item[--] \noindent \textbf{Sport : } football (cinq années), tennis (cinq années), badminton (quatre années), aviron (une année); pratique libre course à pied et cyclisme. 
	\end{itemize}	
	
	
	
	%\vspace{5mm}
	%\vspace{-20pt}
	%\section*{{ \LARGE \color{myblue}Projets}\xfilll[0pt]{0.5pt}}
	%\vspace{-15pt}
	%\vspace{1.5mm}
	
	
	
	%\textbf{\large{TESLA}} (Toulouse European School of Law Association) \hfill {\ \small [2021 - 2022]}
	%\\[-0.05cm]
	%\emph{}%Ensemble de projets menés au sein de l'association étudiante \textbf{TESLA} (UT1), auprès du Président de l'\textbf{ESL} Norbert Angelopoulos.}
	%\\[-0.6cm]	
	%	\begin{itemize}[itemsep = -0.75 mm]
	%		\item[--] \noindent Développement (technologie : \textit{Flutter}) et maintien de l'application mobile : \textbf{TESLA Mobile}. AAAAAAAA DEFINIR TOUT CA TOUT CA
	%		\item[--] \noindent \textbf{Tools:} Quartus Altera, Matlab, AutoCAD, SolidWorks, NGSpice, Xcircuits, MS Office, Salesforce
	%		\item[--] \noindent \textbf{Design:} Figma, Illustrator, Photoshop
	%	\end{itemize} 	
	%\textbf{\large{Application de ``trading''}} \hfill {\ \small [2021]}
	%\\[-0.05cm]
	%\emph{Guide Prof. Meenakshi Gupta, Humanities and Social Sciences Department } \hfill {\ \small \emph {Course Project}}
	%\\[-0.6cm]	
	%	\begin{itemize}[itemsep = -0.75 mm]
	%	\item \noindent Comprehensively assessed the need for \textbf{corporate} inclusion \& diversity and identified the areas of concern
	%	\item \noindent Identified the areas companies are lagging like gender ratio, age diversification, etc to inculcate the diversity
	%	\item \noindent Suggested the effective \textbf{staffing and policy reformations} to rectify the problems prevailing in the firms
	%	\end{itemize}
	
	
	
	
	
	
	
	
	
	
	
	%%%
	% Changer tout ça et faire que ça ressemble plus à un ``recueil de bibliographie'' simple !
	%%%
	\newpage
	\vspace{5mm}
	\vspace{-20pt}
	\section*{{\LARGE \color{myblue}Écrits}\xfilll[0pt]{0.5pt}}
	\vspace{-15pt}
	\vspace{1.5mm}
	
	
	
	%\underline{\textbf{\large{Droit}}} % METTRE UNE PETITE DESCRIPTION POUR CHAQUE ARTICLE TOUT CA TOUT CA (?)
	%\\[-0.6cm]	
	%	\begin{itemize}[itemsep = -0.75 mm]
	%		\item[--] \noindent \textbf{Travaux préparatoires} à un projet d'étudiants pour étudiants. Le but étant de fédérer des étudiants de tous horizons derrière un projet commun, ou plutôt une ``expérience''. En résultera la rédaction d'une sorte de \textit{summa} estudiantine.
	%			\begin{itemize}
	%				 \item[--] \small{\textbf{Projet 0}, ensemble de notes relatives à la structure et au fonctionnement du projet. (38 pages)\hfill {\ \small [2021]}}
	%				 \item[] \tab \small{SSSSS}
	%				 \item[--] \small{\textbf{Travaux préparatoires numéro 1}. \hfill {\ \small [En préparation]}}
	%				 \item[] \tab \small{SSSSS}
	%			\end{itemize}
	%		\item[--]	
	%	\end{itemize} 	
	\underline{\textbf{\large{Mathématiques}}}
	%\\[-0.05cm]
	%\emph{Domaines d'intérêts : algèbre générale, analyse réelle, histoire des sciences.}
	\\[-0.6cm]	
		\begin{itemize}[itemsep = -0.75 mm]
			\item[--] \noindent \textit{Ensembles intégraux de cardinal infini}, avec Decan de Chatouville Raphaël. (12 pages)  \hfill {\ \small [2022]}
			\item[--] \noindent \textbf{Compilation} \textit{des cours de Claire Voisin au Collège de France (Cours de géométrie algébrique : 2015 -- 2020)}. (48 pages)  \hfill {\ \small [2022]}
			\item[--] \noindent \textbf{Cours et développements}. \textit{Apprendre en écrivant}. (198 pages)  \hfill {\ \small [2022]}
			%\item[--] \noindent \textbf{Synthèse}, \textit{Surroundings of Fourier analysis: An attempt to do Algebra, From a general mathematical introduction to harmonic analysis. Autonomous research dissertation in (mostly) pure mathematics.} (48 pages)  \hfill {\ \small [2024]}
			%\item[--] \noindent \textbf{Bibliographie}. \textit{Centres d'intérêt} (version I). (8 pages)  \hfill {\ \small [2022]}
	%		\item[--] \noindent \textbf{Article} (en préparation), \textit{Nature des nombres et relations induites sur des structures algébriques : d'Abel-Ruffini aux périodes}. (X pages)  \hfill {\ \small [2021]}
	%			\begin{itemize}
	%				 \item[] \tab \small{[\textit{Temporaire}] Dans cette tentative d'article, un intérêt tout particulier est porté aux représentations intégrales de différents objets. De multiples relations se déduisent entre lesdites représentations intégrales et l'on essaie, dès lors, d'en exhiber les propriétés. Plus précisément, la considération portée sur les intégrales abéliennes s'inscrit dans le champ de la théorie des nombres transcendants.}
	%			\end{itemize}	
	%		\item[--] \noindent \textbf{Commentaire} et notes d'apprentissage de l'article de Jean-Louis Colliot-Thélène, \textit{L'arithmétique des variétés rationnelles}, Annales de la Faculté des Sciences de Toulouse, Vol. I, n° 3, 1992, page 295-336. (6 pages manuscrites)  \hfill {\ \small [2022]}	
	%		\item[--] \noindent \textbf{Commentaire} et notes d'apprentissage de \textit{L'arithmétique des variétés rationnelles}, Annales de la Faculté des Sciences de Toulouse, Vol. I, n° 3, 1992, page 295-336. (6 pages manuscrites)  \hfill {\ \small [2022]}	
			\item[--] \noindent \textbf{Traduction} de \textit{Topology without tears} de Sidney A. Morris. (Chapitres 1 et 2)  \hspace*{\fill} {\ \small [2021]}
	%			\begin{itemize}
	%				 \item[] \tab \small{AAAAAAA FINIR (la traduction) et également go faire un résumé}
	%			\end{itemize}
	\text{}\newline
			\item \underline{\textbf{\large{Notes d'apprentissage}}}
				\begin{itemize}
					\item[--] \noindent \textit{\textbf{Synthèse} sur l'approximation uniforme (polynômes de Bernstein, uniformité, densité...) avec Garcia Hugo et Sala Raphaël}. (X p.) \hfill {\ \small [2023] }
					\item[--] \noindent \textit{\textbf{Synthèses} sur les ensembles intégraux et espaces métriques}. (4 p., 19 p., 22p.) \hfill {\ \small [2022] }
					%\item[--] \noindent \textit{\textbf{Calcul} du maximum de la densité de probabilité de présence d'un électron issu d'un Hydrogène}. (8 pages) \hfill {\ \small [2022] }
					
					\item[--] \noindent \textit{\textbf{Exercices} du polycopié Mathématiques : du lycée aux CPGE scientifiques (Lycées Louis-Le-Grand et Henri-IV)}. (50 pages) \hfill {\ \small [2022] }
					
					\item[--] \noindent \textit{\textbf{Commentaire} pour un ami, développements physiques, mathématiques et informatiques utiles pour notre projet de simulateur physique 2d/3d}. (83 pages)  \hfill {\ \small [2022]}
					
					%\item[--] \noindent \textit{\textbf{Notes} de recherches sur l'anatomie computationnelle}. (X pages) \hfill {\ \small [2022--2024] }
	
	
	
					%\item[--] \noindent \textbf{Synthèse (2022 -- 2023)}, \textit{@todolist}. (4 pages)  \hfill {\ \small [2023]}
	
	
	
	
				          %\item[--] \noindent \textit{\textbf{Notes} sur le cours d'Alain Prouté -- Logique catégorique}. (9 pages) \hfill {\ \small [??/2022]}
	%			          \item[--] \noindent \textit{Techniques téléologiques}, étude. (5 pages) \hfill {\ \small [??/2022]}
	
	
				          %\item[--] \noindent \textit{Norme et distance}. (2 pages) \hfill {\ \small [04/2022]}
				          %\item[--] \noindent \textit{Continuité(s)}. (2 pages) \hfill {\ \small [04/2022]}
	
	
				          \item[--] \noindent \textit{\textbf{Commentaire} pour un ami et explications d'un article de Don Zagier, From quadratic functions to modular functions}. (23 pages) \hfill {\ \small [02/2022]}
				          \item[--] \noindent \textit{Suites et transformations de polygones réguliers}, début d'étude. (10 pages) \hfill {\ \small [02/2022]}
				          \item[--] \noindent \textit{Indicatrice d'Euler}. (7 pages) \hfill {\ \small [02/2022]}
				          \item[--] \noindent \textit{$k$-formes différentielles (introduction)}, avec Pierre-Louis Frabel. (27 pages) \hfill {\ \small [01/2022]}
				          \item[--] \noindent \textit{Intégrale du 26 mai 2021}. (9 pages) \hfill {\ \small [01/2022]}
				          \item[--] \noindent \textit{\textbf{Synthèse (2020 -- 2021)}, Zibaldone, un peu de Mathématiques}. (32 pages)  \hfill {\ \small [2021]}
					      \item[--] \noindent \textit{\textbf{Commentaire} pour un ami et explications d'un article d'Henri Poincaré, \textit{Sur les hypothèses fondamentales de la géométrie}, Bulletin de la S.M.F., tome 15 (1887), page 203-216}. (14 pages)  \hfill {\ \small [2018]}	
						  \item[--] \noindent \textit{Digressions mathématiques}, notes manuscrites ((environ) 3000 pages) \hfill {\ \small [2010 -- 2017]}
				\end{itemize}
	%			\begin{itemize}
	%				 \item[] \tab \small{Ébauche d’un commentaire et de quelques explications de \textit{Sur les hypothèses fondamentales de la géométrie} de Poincaré. Analyse linéaire du texte. Il n'y avait aucune prétention dans l'analyse si ce n'est de réussir à cerner quelques objets mathématiques et de partager ma passion.\\ \tab \underline{Mots clefs} : \textit{axiome des parallèles}; \textit{géodésique}; \textit{surfaces du second ordre}; \textit{quadrique de dimension 3}; \textit{classification des surfaces fermées}; \textit{ligne génératrice}; \textit{rapport anharmonique}; \textit{seconde forme fondamentale}; \textit{groupe continu}; \textit{crochet de Poisson}; \textit{mémoire de Riemann}.}
	%			\end{itemize}
		\end{itemize}
	\text{}\newline
	%\underline{\textbf{\large{Littérature}}}
	%\\[-0.05cm]
	%\emph{Domaines d'intérêts : littérature antique et médiévale (notamment italienne).}
	%\\[-0.6cm]	
	%	\begin{itemize}[itemsep = -0.75 mm]
	%		\item[--] \noindent \textbf{Commentaires et critiques du rapport d'étude Zellidja}. \textit{Dante, Ses traces en Toscane et quelques aspects de son oeuvre. Version annotée}. (édition augmentée de X remarques) \hfill {\ \small [En préparation]}
	%			\begin{itemize}
	%				 \item[] \tab \small{SSSSSS}
	%			\end{itemize}	
	%			\begin{itemize}
	%				 \item[] \tab \small{Ce rapport n'aurait pas été possible sans le soutien d'une bourse de voyage et d'étude octroyée par la fondation Zellidja. Au terme d'un mois en Toscane, ayant permis de découvrir la culture italienne et d'accumuler souvenirs et références, la construction de ce rapport a commencé. \\ \tab \underline{Mots clefs} : \textit{Langage, papauté, empire, guelfes, gibelins, institutions, récit historique et récit mythique, éducation et transmission du savoir, italianità, tourisme florentin, Divine Comédie, symbolique et histoire, massification des écrits, littérature italienne.}} %AAAAAAAA FINIR
	%			\end{itemize}			
	%	\end{itemize}
	\underline{\textbf{\large{Divers}}}
	\\[-0.6cm]	
	
		\begin{itemize}[itemsep = -0.75 mm]
			\item[--] \noindent \textbf{Rapport étudiant}, \textit{Devenir chercheur en Mathématiques}. (27 pages). \hfill {\ \small [2023]}
			\item[--] \noindent \textit{Détection de particules chargées dans un modèle simplifié de chambre à bulles}, avec Decan de Chatouville Raphaël. (21 pages, 34 pages). \hfill {\ \small [2022]}
			\item[--] \noindent \textbf{Présentation, Droit comparé}, \textit{Azione in giustizia; Diritto civile italiano / Droit civil français}. \hfill {\ \small [2021]}
			\item[--] \noindent \textbf{Article synthétique}, \textit{Renovation of residential heritage in European context}, avec Christina Mardavani et Darío Fazli Khan Moreno. \hfill {\ \small [2021]}
	%			\begin{itemize}
	%				 \item[] \tab \small{SSSSSSSSS}
	%			\end{itemize}		
			\item[--] \noindent \textbf{Notes éparpillées I}, \textit{Mathématiques, philosophie et droit}. (256 pages) \hfill {\ \small [2021]}
			\item[--] \noindent \textbf{Synthèse d'un exposé}, \textit{Social et Travail, relations et organisations}. (12 pages) \hfill {\ \small [2020]}
	%			\begin{itemize}
	%				 \item[] \tab \small{On s'est proposé d'étudier, de manière très succincte, quatre éléments (plus ou moins) caractéristiques des notions émanant du contraste ``social'' / ``travail'' : la déclaration de l'Organisation Internationale du Travail (\textit{sur la justice sociale pour une mondialisation équitable}), un extrait du livre \textit{Flatland}, les colonnes d'Hercule et l'Italie communale. Le travail consistait en diverses tentatives de connexions de ces différents éléments. À cet effet, a été repris (pages 71-97, \textit{Rapport d'étude Zellidja}) ainsi que développé un petit modèle cyclique tentant de donner une certaine approximation du développement des sociétés. Ce modèle se construit essentiellement ainsi : légitimité $\rightarrow$ civilisation $\rightarrow$ besoin $\rightarrow$ cohérence $\rightarrow$ mythe $\rightarrow$ choix $\rightarrow$ légitimité $\rightarrow$ ...\\ \tab \underline{Mots clefs} : \textit{Organisation Internationale du Travail, Flatland, colonnes d'Hercule, Italie communale, LCBCMC, langage, interprétation, État, Droit, Histoire, société, travail, justice sociale, écoumène, terra cognita, homo juridicus, homo economicus, ars dictaminis, Empire, Église, popolo, hiérarchie.}}
	%			\end{itemize}
			\item[--] \noindent \textbf{Rapport d'étude Zellidja}. \textit{Dante, Ses traces en Toscane et quelques aspects de son oeuvre}, sous le tutorat Zellidja de Monsieur Dadillon (Laurent). (125 pages) \\ \textit{(Seconde version, augmentée de 112 annotations.)} \hfill {\ \small [Juillet 2019 -- Décembre 2019]}		
			\item[--] \noindent \textbf{Travail Pratique Encadré}, \textit{Peut-on résoudre un problème physique en s'inspirant du vivant ?}, avec Enjalbert Tadou et Muteau Achille. (38 pages) \hfill {\ \small [Septembre 2018 -- Février 2019]}
	%			\begin{itemize}
	%				 \item[] \tab \small{Dans ce travail, nous nous sommes initiés au monde de la recherche et l'avons découvert. Un des buts était d'utiliser les possibilités permises par l'informatique et de les combiner à des systèmes inspirés du vivant afin de résoudre des problèmes physiques. L'objectif principal était alors de s'intéresser aux mouvement des planètes et d'essayer de comprendre leur trajectoire. En pratique, ce fut un sacré échec. En théorie, ce fut tout à fait intéressant. \\ À cela l'on peut ajouter que nous avions cherché à nous poser une question bien plus générale : \textit{pour un problème $X$ quelconque, peut-on à partir d'un ensemble de concepts $C$ le résoudre ? Si oui, en combien de temps ? Et, existe-t'il une manière optimale ?} Les tentatives de réponses proposées furent très, en général, illusoires. Toujours est-il que, quoique naïves, elles furent drôles et intéressantes. \\ \tab \underline{Mots clefs} : \textit{Mathématiques, Physique, Science et Vie de la Terre, réseaux neuronaux, bio-mimétisme, deep learning, simulation informatique, complexité paramétrée, lois de Kepler, ellipse.}}
	%			\end{itemize}		
		\end{itemize}
	
	
	
	
	
	
	
	\newpage
	\vspace{5mm}
	\vspace{-20pt}
	\section*{{\LARGE \color{myblue}Écrits (blog)}\xfilll[0pt]{0.5pt}}
	\vspace{-15pt}
	\vspace{1.5mm}
	
	
	\underline{\textbf{\large{Mathématiques. Informatique. Linguistique. Formalisation. (M.I.L.F.)}}}
	\\[-0.6cm]	
	
	\begin{itemize}[itemsep = -0.75 mm]
		\item[--] \noindent ... \hfill {\ \small [2024]}
		\item[--] \noindent 5 -- Ben Eater (et cie), ces héros…
		\item[--] \noindent 4 -- L’entrée au Fab Lab.
		\item[--] \noindent 3 -- Cori et Lascar (débuts).
		\item[--] \noindent 2 -- Premier compilateur miniature.
		\item[--] \noindent 1 -- Réinventer la roue.
		\item[--] \noindent 0 -- Objectifs initiaux. \hfill {\ \small [2023]}
	\end{itemize}
	
	
	
	
	
	
	
	
	
	
	
	
	
	
	
	
	
	
	
	
	
	
	
	
	
	
	
	
	
	
	
	
	
	\newpage
	\vspace{5mm}
	\vspace{-20pt}
	\section*{{\LARGE \color{myblue}Exposés oraux}\xfilll[0pt]{0.5pt}}
	\vspace{-15pt}
	\vspace{1.5mm}
	
	
		\begin{itemize}[itemsep = -0.75 mm]
			\item[--] \noindent \textbf{Introduction aux phénomènes de régularisation, apprendre à faire attention}, présentations entre amis (7 minutes). \hfill {\ \small [01/02/2023]}
			\item[--] \noindent \textbf{Détection de particules chargées dans une chambre à bulles} avec Raphaël Decan de Chatouville, soutenance projet de L1 S1 PS (40 minutes). \hfill {\ \small [10/01/2023]}		
			\item[--] \noindent \textbf{Espaces de Hilbert à noyau reproduisant (introduction et calculs sur l'exemple de l'espace de Hardy $H^2$)} (40 minutes; désastreux). \hfill {\ \small [08/04/2022]}
			\item[--] \noindent \textbf{Panorama introductif à l'arithmétique} (1 heure 15 minutes). \hfill {\ \small [26/03/2022]}
			\item[--] \noindent \textbf{Ouvertures sur les Mathématiques} (3 heures). \hfill {\ \small [2020]}
		\end{itemize}
	
	
	
	
	
	
	
	
	
	
	
	
	
	
	
	
	%%%
	% Changer tout ça et faire que ça ressemble plus à un ``recueil de bibliographie'' simple !
	%%%
	\newpage
	\vspace{5mm}
	\vspace{-20pt}
	\section*{{\LARGE \color{myblue}Programmation}\xfilll[0pt]{0.5pt}}
	\vspace{-15pt}
	\vspace{1.5mm}
	
	
	
	%\underline{\textbf{\large{Droit}}} % METTRE UNE PETITE DESCRIPTION POUR CHAQUE ARTICLE TOUT CA TOUT CA (?)
	%\\[-0.6cm]	
	%	\begin{itemize}[itemsep = -0.75 mm]
	%		\item[--] \noindent \textbf{Travaux préparatoires} à un projet d'étudiants pour étudiants. Le but étant de fédérer des étudiants de tous horizons derrière un projet commun, ou plutôt une ``expérience''. En résultera la rédaction d'une sorte de \textit{summa} estudiantine.
	%			\begin{itemize}
	%				 \item[--] \small{\textbf{Projet 0}, ensemble de notes relatives à la structure et au fonctionnement du projet. (38 pages)\hfill {\ \small [2021]}}
	%				 \item[] \tab \small{SSSSS}
	%				 \item[--] \small{\textbf{Travaux préparatoires numéro 1}. \hfill {\ \small [En préparation]}}
	%				 \item[] \tab \small{SSSSS}
	%			\end{itemize}
	%		\item[--]	
	%	\end{itemize} 	
	
	
	\underline{\textbf{\large{2023}}}
	\\[-0.6cm]	
	
	\begin{itemize}[itemsep = -0.75 mm]
			\item[--] \noindent Introduction à l'écosystème \textbf{Vulkan} / \textbf{GLFW}. (C++)
			\item[--] \noindent \textbf{TinyComp 1}, découverte des principes de base d'un compilateur. (Python)
	\end{itemize}
	
	
	
	\underline{\textbf{\large{2022}}}
	%\\[-0.05cm]
	%\emph{Domaines d'intérêts : algèbre générale, analyse réelle, histoire des sciences.}
	\\[-0.6cm]	
		\begin{itemize}[itemsep = -0.75 mm]
			\item[--] \noindent \textbf{Intégration d'équations différentielles} par l'utilisation de schémas numériques (Runge-Kutta 4, \textit{boundary value problem}, méthode de shooting non linéaire). (Fortran)
			\item[--] \noindent \textbf{Équation de Schrödinger polyélectronique} et bases de \textbf{chimie quantique computationnelle} (prémisses de méthodes d'\textbf{Hartree-Fock}). (Python)
			\item[--] \noindent \textbf{Calcul, atomistique}, maximum de la densité de probabilité de présence d'un électron issu d'un Hydrogène. (Python)
			\item[--] \noindent \textbf{The Natural Number Game} \textit{(version 1.3.3, \href{https://www.ma.imperial.ac.uk/~buzzard/xena/natural_number_game/}{imperial.ac.uk})}. (Lean)
			\item[--] \noindent \textbf{Simulations de trajectoires} \textit{nécessitant l'implémentation de méthodes élémentaires d'approximation de solutions d'équations différentielles non linéaires}. (C, OpenGL, glut)
			\item[--] \noindent \textbf{Introduction à l'analyse de texte}, \textit{Étude textuelle de 4220 arrêts du Conseil d'État. Thème : responsabilité de la puissance publique (droit et contentieux administratif). Extension du corpus à 113 années de jurisprudence (Gallica)}. (Python)
	
		\end{itemize}
	
	
	
	\underline{\textbf{\large{2021}}}
	\\[-0.6cm]	
	
		\begin{itemize}[itemsep = -0.75 mm]
			\item[--] \noindent \textbf{Calcul symbolique}, \textit{Systématisation et calcul d'intégrales}. (Python)
			\item[--] \noindent \textbf{Thèorème des deux carrés de Fermat}, \textit{Généralisation et recherche d'un critère similaire (dans $\mathbb{Z}[\zeta_3]$) dans diverses configurations}. (Python)
		\end{itemize}
	
	\underline{\textbf{\large{2020}}}
	\\[-0.6cm]	
	
		\begin{itemize}[itemsep = -0.75 mm]
			\item[--] \noindent \textbf{Constellation}, \textit{Rendu graphique aléatoire de constellations}. (Python)
		\end{itemize}
	
	\underline{\textbf{\large{2019}}}
	\\[-0.6cm]	
	
		\begin{itemize}[itemsep = -0.75 mm]
			\item[--] \noindent \textbf{Javaquarium}, \textit{Implémentation d'un aquarium (avec fichiers de sauvegarde etc...)}. (Python)
		\end{itemize}
	
	\underline{\textbf{\large{2018}}}
	\\[-0.6cm]	
	
		\begin{itemize}[itemsep = -0.75 mm]
			\item[--] \noindent \textbf{FakePeano}, \textit{Implémentation de l'arithmétique de Peano}. (Python)
		\end{itemize}
	
	\underline{\textbf{\large{2017}}}
	\\[-0.6cm]	
	
		\begin{itemize}[itemsep = -0.75 mm]
			\item[--] \noindent \textbf{KeyProject}, \textit{Gestionnaire de jeux de clefs avec fichiers de configuration etc...} (C, Java)
			\item[--] \noindent \textbf{FakeFile}, \textit{Micro gestionnaire de fichier et système de livre virtuel}. (C++)
			\item[--] \noindent \textbf{TSS}, \textit{Micro moteur 2D avec gestion de collisions}. (Java)
			\item[--] \noindent \textbf{Craft}, \textit{Système complet de craft (7 items et 28 recettes)}. (C++)
		\end{itemize}
	
	
	\underline{\textbf{\large{2016}}}
	\\[-0.6cm]	
	
		\begin{itemize}[itemsep = -0.75 mm]
			\item[--] \noindent \textbf{mini-Bibliothèque}, \textit{Manipulations vectorielles}. (C)
			\item[--] \noindent \textbf{Petite voiture}, \textit{support mobile capable de se déplacer de manière autonome dans un environnement (résultats : peu concluants)}. (Arduino)
		\end{itemize}
	
	
	
	
	
	
	
	
	
	
	
	
	\newpage
	\vspace{5mm}
	\vspace{-20pt}
	\section*{{\LARGE \color{myblue}Cours suivis}\xfilll[0pt]{0.5pt}}
	\vspace{-15pt}
	\vspace{1.5mm}
	
	\text { }
	\newline
	
	
	
	\underline{\textbf{\large{Droit (2020 -- 2022)}}}
	\\[-0.6cm]	
		\begin{itemize}[itemsep = -0.75 mm]
			\item[--] \noindent \textbf{Histoire du droit et des institutions} : SUTRA Romy, HENNING Jérôme.
			\item[--] \noindent \textbf{Droit international et européen} : BLANC Didier, BETAILLE Julien.
			\item[--] \noindent \textbf{Droit public et administratif} : WANDA Mastor, CARPENTIER Mathieu, TOUZEIL-DIVINA Mathieu, THERON Sophie.
			\item[--] \noindent \textbf{Droit des finances publiques et droit fiscal} : VIOLA André, DUSSART Vincent.
			\item[--] \noindent \textbf{Droit privé} : RASS MASSON Lukas, JEAN Séverin, POUMAREDE Matthieu, NICOD Marc.
			\item[--] \noindent \textbf{Droit des affaires} : PICOD Nathalie.
			\item[--] \noindent \textbf{Droit des sociétés } : SEGONDS Marc. 
			\item[--] \noindent \textbf{Droit pénal et procédure pénale} :  BOTTON Antoine.
			\item[--] \noindent \textbf{Droit italien} : POSOCCO Laurent, PERLO Nicoletta, MONTERMINI Giovanna, CORBION Lycette, MOTRONI Raimondo, BRIGANTE Vinicio, BOCCAGNA Salvatore, PERLO Filippo, RENON Paolo, TUCCILLO Fabiana, SALVATORE Barbara, D'ACUNTO Luciana.
			\item[--] \noindent \textbf{Anglais} : DOUSSOT Audrey, GOUSSET Tiphaine.
		\end{itemize}
	
	
	
	\text{ } \newline
	
	\underline{\textbf{\large{Parcours Spécial mention Mathématiques (2022 -- )}}}
	\\[-0.6cm]	
		\begin{itemize}[itemsep = -0.75 mm]
			\item[--] \noindent \textbf{Mathématiques} : BERNARDARA Marcello.
			\item[--] \noindent \textbf{Physique} : DEHEUVELS Sébastien.
			\item[--] \noindent \textbf{Chimie} : CUNY Jérôme.
			\item[--] \noindent \textbf{Outils mathématiques} : HENNEQUIN Théo.
		\end{itemize}
	
	\newpage
	\vspace{5mm}
	\vspace{-20pt}
	\section*{{\LARGE \color{myblue}Travail}\xfilll[0pt]{0.5pt}}
	\vspace{-15pt}
	\vspace{1.5mm}
	
	\text { }
	\newline
	
	
	
	
	\begin{itemize}[itemsep = -0.75 mm]
		\item \textbf{Job étudiant}, équipier polyvalent, McDo (juin 2022).
	\end{itemize}
	\end{document}
